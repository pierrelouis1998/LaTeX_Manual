\documentclass[11pt]{article}				% autres choix : book, report

\usepackage[utf8]{inputenc}					% gestion des accents (source)
\usepackage[T1]{fontenc}					% gestion des accents (PDF)
\usepackage[french]{babel}				    % gestion du français

\usepackage{textcomp}						% caractères additionnels
\usepackage{mathtools,amssymb,amsthm}		% packages de l'AMS + mathtools
\usepackage{lmodern}						% police de caractère
\usepackage{stmaryrd}						% symboles supplémentaires
\usepackage{csquotes}
\usepackage{empheq}							% pour encadrer.

\usepackage{geometry}						% gestion des marges
\geometry{top=3cm, bottom=2cm, left=2.5cm, right=2.5cm}

\usepackage{graphicx}						% gestion des images
\usepackage{xcolor}							% gestion des couleurs
\usepackage{array}							% gestion améliorée des tableaux
\usepackage{multirow}						% gestion améliorée des colonnes

\usepackage[framemethod=tikz]{mdframed}		% mise en page
\usepackage{calc}							% syntaxe naturelle pour les calculs
\usepackage[pagestyles]{titlesec}			% pour les sections
\usepackage{titletoc}						% pour la table des matières
\usepackage{fancyhdr}						% pour les en-têtes
\usepackage{wrapfig}

\usepackage{tikz, pgf}
\usepackage{tikz-3dplot}
\usepackage{tkz-euclide}
\usepackage{pgfplots}						% tracer des courbes
\usepackage{pgfplotstable}

\usepackage{hyperref}						% permet de mettre des url cliquables
\usepackage{listings}						% permet de mettre du code

%\usetkzobj{all}
\usetikzlibrary{calc}
\pgfplotsset{compat=1.7}

\title{\textbf{Fiche Récapitulative \\ des commandes \LaTeX}
\author{Le KI'018}
\date{}
}

\newpagestyle{main}{
  \sethead{\sectiontitle}                       
	{}                                          
	{\quad$|$\quad\thepage}     				
}
\pagestyle{main}

\begin{document}
\maketitle

\begin{figure}[h]
\begin{center}
\includegraphics[scale=0.7]{moyen.png}
\end{center}
\end{figure}

\section*{Préambule}

Cette fiche a pour but de rappeler les différentes commandes \LaTeX \ présentées pendant la formation du KI'022. Lorsqu'une commande est présentée ce qui est en italique est ce qui peut être modifié. Le reste est la commande à proprement parler. \\
Enfin, n'oublie pas, cher lecteur avide de savoir, que \textbf{Google est ton ami !}

\tableofcontents



\newpage



\section{La base d'un document \LaTeX}


\noindent Pour débuter un document \LaTeX \ il faut utiliser un certain nombre de commandes et inclure des paquets\footnote{Le template fourni lors de la formation regroupe les paquets les plus utilisés en pratique.}. Voici la liste des commandes les plus fréquentes : \\

\noindent
\verb?\documentclass[?\emph{options}\verb?]{?\emph{type}\verb?}? \\
\verb?\usepackage[?\emph{options}\verb?]{?\emph{nom}\verb?}? \\
~\\ 
\verb?\title{?\emph{titre}\verb?}? \\
\verb?\author{?\emph{auteur}\verb?}? \\
\verb?\date{?\emph{date}\verb?}? \\
~\\ 
\verb?\begin{document}? \\
\verb?\maketitle? \\
~\\ 
\verb?\tableofcontents? \\
\verb?\part{?\emph{titre de la partie}\verb?}? \\
\verb?\section{?\emph{titre de la section}\verb?}? la numérotation est évitée avec : \verb?\section*{?\emph{titre de la section}\verb?}? \\
\verb?\subsection{?\emph{titre de la sous-section}\verb?}? s'étend en \verb?\subsubsection{?\emph{titre de la sous-sous-section}\verb?}? \\
\verb?\paragraph{?\emph{titre du paragraphe}\verb?}? \\
~\\ 
\verb?\newline? ou \verb?\\? \\
\verb?\newpage?\\
~\\ 
\verb?\end{document}? \\



\section{Les commandes classiques de mise en page}


\noindent Pour modifier le texte, il y a les commandes : \\

\begin{tabular}{lcl}

\verb?\textbf{?\emph{texte}\verb?}? & $\rightarrow$ & \textbf{texte} \\
\verb?\textit{?\emph{texte}\verb?}? & $\rightarrow$ & \textit{texte} \\
\verb?\texttt{?\emph{texte}\verb?}? & $\rightarrow$ & \texttt{texte} \\
\verb?\underline{?\emph{texte}\verb?}? & $\rightarrow$ & \underline{texte} \\
\verb?\emph{?\emph{texte}\verb?}? & $\rightarrow$ & \emph{texte} \\
\verb?\textsc{?\emph{texte}\verb?}? & $\rightarrow$ & \textsc{texte} \\
\verb?\fbox{?\emph{texte}\verb?}? & $\rightarrow$ & \fbox{texte} \\

\end{tabular}\\


\noindent Les listes à puces (respectivement numérotées) se font avec les commandes suivantes : \\

\begin{tabular}{lcl}

\verb?\begin{itemize}? & ou & \verb?\begin{enumerate}? \\
\indent \verb?\item? \emph{texte} & ou & \indent \verb?\item? \emph{texte} \\
\indent \verb?\item? \emph{texte} & ou & \indent \verb?\item? \emph{texte} \\
\verb?\end{itemize}? & ou & \verb?\end{enumerate}? \\

\end{tabular} \\


\newpage


\noindent Les citations se font de la façon suivante : \\
\verb?\begin{quote}? \\
\emph{texte} \\
\verb?\end{quote}? \\

\noindent Et pour des citations de plusieurs lignes : \\
\verb?\begin{quotation}? \\
\emph{texte} \\
\verb?\end{quotation}? \\


\noindent La commande \verb?\footnote{?\emph{texte}\verb?}? crée une note en bas de page. \\


\noindent La position du texte sur la page se modifie avec les commandes : \\

\begin{tabular}{lll}

\verb?\begin{flushleft}? & \verb?\begin{center}? & \verb?\begin{flushright}? \\
texte & texte & texte \\
\verb?\end{flushleft}? & \verb?\end{center}? & \verb?\end{flushright}? \\

\end{tabular}



\section{Des commandes plus avancées}


\noindent Grâce aux commandes \verb?\label{?\emph{label}\verb?}? et \verb?\ref{?\emph{label}\verb?}? \LaTeX \ fait automatiquement des références numérotées à la section où se trouve \verb?\label{?\emph{label}\verb?}?. \\


\noindent Pour inclure une image il faut le paquet : \verb?\usepackage{graphicx}? et ensuite procèder ainsi : \\
\verb?\begin{figure}[?\emph{option}\verb?]? \\
\verb?\begin{center}? \\
\verb?\includegraphics[scale=? \emph{fraction} \verb?]{?\emph{chemin d'accès à l'image}\verb?}? \\
\verb?\caption{?\emph{légende}\verb?}? \\
\verb?\end{center}? \\
\verb?\end{figure}? \\


\noindent La syntaxe pour faire des tableaux n'est pas très compliquée\footnote{Elle est même très similaire à celle des matrices, voir \ref{matrice}.} mais ce lien permet d'en créer rapidement : http://www.tablesgenerator.com \\


\noindent Enfin pour créer une bibliographie il faut procéder ainsi :
\begin{itemize}
\item Un fichier \emph{bibliographie.bib} ;
\item La commande \verb?\cite{?\emph{nom}\verb?}? qui permet de citer une référence ;
\item Les commandes \verb?\bibliographystyle{?\emph{plain}\verb?}? et \verb?\bibliography{?\emph{bibliographie}\verb?}? qui permettent d'afficher la bibliographie.
\end{itemize}

\noindent Les entrées du fichier \emph{bibliographie.bib} se présentent sous la forme :\\
@article\verb?{?\emph{nom}, \\
title=\verb?{?\emph{titre}\verb?}?,\\
author=\verb?{?\emph{auteur}\verb?}?,\\
\emph{type}\verb?={?\emph{titre global}\verb?}?,\\
year=\verb?{?\emph{année}\verb?}?\\
\verb?}?


\newpage



\section{Les commandes mathématiques}


\noindent Le concept est toujours le même, il faut des paquets : \verb?\usepackage{mathtools}? et \verb?\usepackage{amssymb}?. Ensuite il faut définir un environnement mathématiques par \verb?$?\emph{nombres et equations}\verb?$? pour le mettre dans le texte, \verb?$$?\emph{nombres et equations}\verb?$$? pour le mettre centré, à la ligne\footnote{Il en existe d'autre, plus complexe, avec la possibilité de numéroter les équations, voir les liens en \ref{lien}.}. \\


\noindent Voici quelques commandes utiles dans cette environnement : \\

\begin{tabular}{lcl}

\verb?\frac{?\emph{num}\verb?}{?\emph{den}\verb?}? & $\rightarrow$ & $\frac{\emph{num}}{\emph{den}}$ \\
\emph{base}\verb?^{?\emph{exposant}\verb?}? & $\rightarrow$ & $base^{exposant}$ \\
\emph{base}\verb?_{?\emph{indice}\verb?}? & $\rightarrow$ & $base_{indice}$ \\
\verb?\sum_{?\emph{bas}\verb?}^{?\emph{haut}\verb?}?\emph{terme} & $\rightarrow$ & $\sum_{bas}^{haut} terme$ \\ 
\verb?\prod_{?\emph{bas}\verb?}^{?\emph{haut}\verb?}?\emph{terme} & $\rightarrow$ & $\prod_{bas}^{haut} terme$ \\
\verb?\sqrt{?\emph{expression}\verb?}? & $\rightarrow$ & $\sqrt{expression}$ \\
\verb?\sqrt[?\emph{n}\verb?]{?\emph{expression}\verb?}? & $\rightarrow$ & $\sqrt[n]{expression}$ \\
\verb?\lim_{?\emph{expression}\verb?}? & $\rightarrow$ & $\lim_{expression}$ \\
\verb?\int_{?\emph{bas}\verb?}^{?\emph{haut}\verb?}?\emph{terme} & $\rightarrow$ & $\int_{bas}^{haut} terme$ \\

\end{tabular}\\


\noindent Et en complément quelques symboles mathématiques fréquents : \\

\begin{tabular}{lcl}

\verb?\forall? & $\rightarrow$ & $\forall$ \\
\verb?\exists? & $\rightarrow$ & $\exists$ \\
\verb?\in? & $\rightarrow$ & $\in$ \\
\verb?\to? & $\rightarrow$ & $\to$ \\
\verb?\infty? & $\rightarrow$ & $\infty$ \\
\verb?\partial? & $\rightarrow$ & $\partial$ \\
\verb?\mathbb{?\emph{R}\verb?}? & $\rightarrow$ & $\mathbb{R}$ \\
\verb?\mathcal{?\emph{R}\verb?}? & $\rightarrow$ & $\mathcal{R}$ \\
\verb?\mathbf{?\emph{R}\verb?}? & $\rightarrow$ & $\mathbf{R}$ \\

\end{tabular} \\


\noindent Et voici comment faire les lettres greques, minuscules et majuscules :
\begin{tabular}{lcl}

\textbackslash\emph{alpha} & $\rightarrow$ & $\alpha$ \\
\textbackslash\emph{Omega} & $\rightarrow$ & $\Omega$ \\

\end{tabular} \\


\noindent Voici enfin les commandes nécessaires à l'écriture de matrice\footnote{Il faut être dans un environnement mathématique, bien sûr.}\label{matrice} : \\

\begin{tabular}{lcl}

\verb?\begin{matrix}? & & \\

$\begin{matrix}
	 & a & \verb?&? & b & \verb?\\? \\
	 & c & \verb?&? & d & \verb?\\? \\
\end{matrix}$ & $\rightarrow$ & $\begin{matrix}
	a & b \\
	c & d \\
\end{matrix}$ \\

\verb?\end{matrix}? & &

\end{tabular} \\

\noindent Il est possible d'avoir des matrices avec parenthèses en remplaçant \emph{matrix} par \emph{pmatrix}; pour avoir des crochets : \emph{bmatrix}; et pour des barres : \emph{vmatrix}.

\noindent Pour remplir une matrice de point : \verb?\cdots? donne $\cdots$ ou encore, \emph{cdots} peut être remplacé par \emph{vdot}\footnote{Point verticaux $\vdots$} ou \emph{ddots}\footnote{Point diagonaux $\ddots$}. \\


\noindent Un système d'équation se fait simplement comme un tableau encadré de \verb?\left\{? et \verb?\right?.


\newpage



\section{Complément}


\subsection{Inclure du code}


\noindent Il est également possible d'inclure du code dans un fichier \LaTeX \ grâce au paquet \verb?\usepackage{listings}? et aux commandes suivantes : \\

\noindent \verb?\lstset{language=? \emph{nom de langage} \verb?}? \\
\verb?\begin{lstlisting}? \\
\emph{Le code, tel qu'il serait écrit dans un éditeur de texte classique. Exemple de code Python :} \\
for i in range (0: N) : \\
\indent X[i] = methode(X[i - 1]) \\
\verb?\end{lstlisting}? \\

\noindent Cela va inclure le code ainsi :

\lstset{language=c++}
\begin{lstlisting}

// New function.
int function(int x) {
    return x * 10;
}

for (int i = 0; i < 10; i++) {
	int j = function(i);
	std::cout << j << std::endl;
}

\end{lstlisting}

\noindent Il existe de nombreuses options pour ajouter la coloration syntaxique, encadrer le code, numéroter les lignes, mais les liens fournis en \ref{lien} expliquent tous ces détails.


\subsection{Liens utiles}\label{lien}


\noindent Voici quelque lien utile pour chercher des commandes, des options ou des précisions supplémentaires :
\begin{itemize}
	\item \url{https://www.google.com}
	\item \url{tex.stackexchange.com}
	\item \url{https://fr.wikibooks.org/wiki/LaTeX}
	\item \url{https://openclassrooms.com/courses/redigez-des-documents-de-qualite-avec-latex}
	\item \url{https://fr.sharelatex.com}
	\item \url{https://upont.enpc.fr/ponthub/autres}
\end{itemize}


\subsection{Logiciel conseillé}


\noindent ShareLaTeX est une solution pratique pour s'éviter l'installation de \LaTeX \ et faire des projets à plusieurs\footnote{Deux maximum sur la version gratuite.}, néanmoins, il peut être utile d'avoir \LaTeX \ sur son propre ordinateur et donc d'avoir un logiciel de traitement de texte adapté : Sublime Text \footnote{Sublime Text doit être configuré pour gérer \LaTeX.} et TexMaker sont de bons logiciel testés et approuvés par le KI'018.



\end{document}